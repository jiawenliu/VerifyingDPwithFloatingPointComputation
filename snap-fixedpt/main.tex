\documentclass[a4paper,11pt]{article}
\usepackage[utf8]{inputenc}
\usepackage[utf8]{inputenc}%Packages
\usepackage[T1]{fontenc}
\usepackage{fourier} 
\usepackage[english]{babel} 
\usepackage{amsmath,amsfonts,amsthm} 
\usepackage{lscape}
\usepackage{geometry}
\usepackage{amsmath}
\usepackage{algorithm}
\usepackage{algorithmic}
\usepackage{amssymb}
\usepackage{amsfonts}
\usepackage{times}
\usepackage{bm}
\usepackage{mathtools}
\usepackage{ stmaryrd }
\usepackage{ amssymb }
\usepackage{ textcomp }
\usepackage[normalem]{ulem}
% For derivation rules
\usepackage{mathpartir}
\usepackage{color}
\usepackage{a4wide}

\usepackage{stmaryrd}
\SetSymbolFont{stmry}{bold}{U}{stmry}{m}{n}

\newcommand{\distr}{\mathsf{Distr}}
\newcommand{\uniform}{\mathsf{unif}}
\newcommand{\pdf}{\mathsf{pdf}}
\newcommand{\snap}{\mathsf{Snap}}
\newcommand{\fsnap}{\mathsf{Snap}_{\mathbb{F}}}
\newcommand{\rsnap}{\mathsf{Snap}_{\mathbb{R}}}


\newcommand{\pr}[2]{\underset{#1}{\mathsf{Pr}}[#2]}
\newcommand{\projl}{\pi_1}
\newcommand{\projr}{\pi_2}
\newcommand{\supp}{\mathsf{supp}}
\newcommand{\clamp}{\mathsf{clamp}}
\newcommand{\real}{\mathbb{R}}
\newcommand{\samplel}{\xleftarrow{\$}}
\newcommand{\psup}{\mathsf{Sup}}
\newcommand{\sign}{\mathsf{sign}}

\newcommand{\lapmech}{\mathcal{L}}
\newcommand{\laplace}{\mathsf{laplce}}
\newcommand{\round}[1]{\lfloor #1 \rceil}


%for syntax:

%for programs:
\newcommand{\prog}{p}
\newcommand{\fprog}{p_{\mathbb{F}}}
\newcommand{\rprog}{p_{\mathbb{R}}}
\newcommand{\ret}{\mathsf{return}}



%expression
\newcommand{\expr}{e}
\newcommand{\fexpr}{\expr_{\mathbb{F}}}
\newcommand{\rexpr}{\expr_{\mathbb{R}}}

%for smaples:
\newcommand{\bernoulli}{\kw{bernoulli}}

%values
\newcommand{\fval}{c}
\newcommand{\rval}{r}
\newcommand{\valv}{v}
\newcommand{\data}{D}

%variables
\newcommand{\varx}{x}

\newcommand{\fvarx}{x}
\newcommand{\rvarx}{X}


\newcommand{\term}{t}
\newcommand{\etrue}{\kw{true}}
\newcommand{\efalse}{\kw{false}}
% \newcommand{\eflconst}{c}
% \newcommand{\erlconst}{r}
\newcommand{\precision}{\eta}
\newcommand{\floaten}{\kw{fl}}

\newcommand{\err}{err}
\newcommand{\condition}{\Phi}
\newcommand{\edistr}{\mu}

\newcommand{\bigstep}{\Downarrow}
\newcommand{\trsto}{\Rightarrow}


%for environments
\newcommand{\trsenv}{\Theta}

\newcommand{\evlenv}{\Gamma}



\usepackage{stackengine} 

% For Operations
%binary operations
\newcommand{\bop}{*}
\newcommand{\obop}{\stackMath\mathbin{\stackinset{c}{0ex}{c}{0ex}{\text{\footnotesize{$\bop$}}}{\bigcirc}}}

\newcommand{\oexp}{\stackMath\mathbin{\stackinset{c}{0ex}{c}{0ex}{\text{\footnotesize{$\mathsf{e}$}}}{\bigcirc}}}

\newcommand{\oln}{\stackMath\mathbin{\stackinset{c}{0ex}{c}{0ex}{\text{\footnotesize{$\mathsf{ln}$}}}{\bigcirc}}}

\newcommand{\odiv}{\stackMath\mathbin{\stackinset{c}{0ex}{c}{0ex}{\text{\footnotesize{$\div$}}}{\bigcirc}}}
\newcommand{\ubar}[1]{\text{\b{$#1$}}}

%unary operations
\newcommand{\uop}{\circ}
\newcommand{\ouop}{\stackMath\mathbin{\stackinset{c}{0ex}{c}{0ex}{\text{\footnotesize{$\uop$}}}{\bigcirc}}}








% \DeclareSymbolFont{extraup}{U}{zavm}{m}{n}
% \DeclareMathSymbol{\vardiamond}{\mathalpha}{extraup}{87}

\newcommand{\diam}{{\color{red}\diamond}}
\newcommand{\dagg}{{\color{blue}\dagger}}
\let\oldstar\star
\renewcommand{\star}{\oldstar}

\newcommand{\im}[1]{\ensuremath{#1}}

\newcommand{\kw}[1]{\im{\mathtt{#1}}}


\newcommand{\set}[1]{\im{\{{#1}\}}}

\newcommand{\mmax}{\ensuremath{\mathsf{max}}}

%%%%%%%%%%%%%%%%%%%%%%%%%%%%%%%%%%%%%%%%%%%%%%%%%%%%%%%%
% Comments
\newcommand{\omitthis}[1]{}

% Misc.
\newcommand{\etal}{\textit{et al.}}
\newcommand{\bump}{\hspace{3.5pt}}

% Text fonts
\newcommand{\tbf}[1]{\textbf{#1}}
%\newcommand{\trm}[1]{\textrm{#1}}

% Math fonts
\newcommand{\mbb}[1]{\mathbb{#1}}
\newcommand{\mbf}[1]{\mathbf{#1}}
\newcommand{\mrm}[1]{\mathrm{#1}}
\newcommand{\mtt}[1]{\mathtt{#1}}
\newcommand{\mcal}[1]{\mathcal{#1}}
\newcommand{\mfrak}[1]{\mathfrak{#1}}
\newcommand{\msf}[1]{\mathsf{#1}}
\newcommand{\mscr}[1]{\mathscr{#1}}

% Text mode
\newenvironment{nop}{}{}

% Math mode
\newenvironment{sdisplaymath}{
\begin{nop}\small\begin{displaymath}}{
\end{displaymath}\end{nop}\ignorespacesafterend}
\newenvironment{fdisplaymath}{
\begin{nop}\footnotesize\begin{displaymath}}{
\end{displaymath}\end{nop}\ignorespacesafterend}
\newenvironment{smathpar}{
\begin{nop}\small\begin{mathpar}}{
\end{mathpar}\end{nop}\ignorespacesafterend}
\newenvironment{fmathpar}{
\begin{nop}\footnotesize\begin{mathpar}}{
\end{mathpar}\end{nop}\ignorespacesafterend}
\newenvironment{alignS}{
\begin{nop}\begin{align}}{
\end{align}\end{nop}\ignorespacesafterend}
\newenvironment{salignS}{
\begin{nop}\small\begin{align}}{
\end{align}\end{nop}\ignorespacesafterend}
\newenvironment{falignS}{
\begin{nop}\footnotesize\begin{align*}}{
\end{align}\end{nop}\ignorespacesafterend}

% Stack formatting
\newenvironment{stackAux}[2]{%
\setlength{\arraycolsep}{0pt}
\begin{array}[#1]{#2}}{
\end{array}}
\newenvironment{stackCC}{
\begin{stackAux}{c}{c}}{\end{stackAux}}
\newenvironment{stackCL}{
\begin{stackAux}{c}{l}}{\end{stackAux}}
\newenvironment{stackTL}{
\begin{stackAux}{t}{l}}{\end{stackAux}}
\newenvironment{stackTR}{
\begin{stackAux}{t}{r}}{\end{stackAux}}
\newenvironment{stackBC}{
\begin{stackAux}{b}{c}}{\end{stackAux}}
\newenvironment{stackBL}{
\begin{stackAux}{b}{l}}{\end{stackAux}}

%APPENDIX
\newcommand{\caseL}[1]{% \setcounter{equation}{0}

\item \textbf{#1}\newline}

%% \makeatletter
%% \newcommand\definitionname{Lemma}
%% \newcommand\listdefinitionname{Proofs of Lemmas and Theorems}
%% \newcommand\listofdefinitions{%
%%   \section*{\listdefinitionname}\@starttoc{def}}
%% \makeatother



\newtheoremstyle{athm}{\topsep}{\topsep}%
      {\upshape}%         Body font
      {}%         Indent amount (empty = no indent, \parindent = para indent)
      {\bfseries}% Thm head font
      {}%        Punctuation after thm head
      {.8em}%     Space after thm head (\newline = linebreak)
      {\thmname{#1}\thmnumber{ #2}\thmnote{~\,(#3)}
% \addcontentsline{Lemma}{Lemma}
%   {\protect\numberline{\thechapter.\thelemma}#1}
      % \ifstrempty{#3}%
      {\addcontentsline{def}{section}{#1~#2~#3}}%
      % {\addcontentsline{def}{subsection}{\theathm~#3}}
\newline}%         Thm head spec

 \theoremstyle{athm}


% \newtheoremstyle{break}
%   {\topsep}{\topsep}%
%   {\itshape}{}%
%   {\bfseries}{}%
%   {\newline}{}%
% \theoremstyle{break}

%There are some problems with llncs documentcalss, so commenting these out until i find a solution
\newtheorem{thm}{Theorem}

%\spnewtheorem{thm1}[theorem]{Theorem}{\bfseries}{\upshape}
%\newenvironment{Theorem}[1][]{\begin{thm1}\iffirstargument[#1]\fi\quad\\}{\end{thm1}}

 \newtheorem{lem}[thm]{Lemma}
 \newtheorem{conjec}{Conjecture}
 \newtheorem{corr}[thm]{Corollary}
 \newtheorem{defn}{Definition}
 \newtheorem{prop}[thm]{Proposition}
 \newtheorem{assm}[thm]{Assumption}

\newtheorem{Eg}[thm]{Example}
\newtheorem{hypothesis}[thm]{Hypothesis}
\newtheorem{motivation}{Motivation}

% BNF symbols
\newcommand{\bnfalt}{{\bf \,\,\mid\,\,}}
\newcommand{\bnfdef}{{\bf ::=~}}

%% Highlighting
\newcommand{\hlm}[1]{\mbox{\hl{$#1$}}}

%% Provenance modes
\newcommand{\modifrcationProvenance}{{\bf MP}}
\newcommand{\updateProvenance}{{\bf UP}}

%Lemmas
\newcommand{\lemref}[1]{Lemma \ref{#1}} %name and number
\newcommand{\thmref}[1]{Theorem \ref{#1}} %name and number

\renewcommand{\labelenumii}{\theenumii}
\renewcommand{\theenumii}{\theenumi.\arabic{enumii}.}

\usepackage{enumitem}
\setenumerate{listparindent=\parindent}

\newlist{enumih}{enumerate}{3}
\setlist[enumih]{label=\alph*),before=\raggedright, topsep=1ex, parsep=0pt,  itemsep=1pt }

\newlist{enumconc}{enumerate}{3}
\setlist[enumconc]{leftmargin=0.5cm, label*= \arabic*.  , topsep=1ex, parsep=0pt,  itemsep=3pt }

\newlist{enumsub}{enumerate}{3}
\setlist[enumsub]{ leftmargin=0.7cm, label*= \textbf{subcase} \bf \arabic*: }

\newlist{enumsubsub}{enumerate}{3}
\setlist[enumsubsub]{ leftmargin=0.5cm, label*= \textbf{subsubcase} \bf \arabic*: }

\newlist{mainitem}{itemize}{3}
\setlist[mainitem]{ leftmargin=0cm , label= {\bf Case} }


\newenvironment{subproof}[1][\proofname]{%
  \renewcommand{\qedsymbol}{$\blacksquare$}%
  \begin{proof}[#1]%
}{%
  \end{proof}%
}


\newenvironment{nstabbing}
  {\setlength{\topsep}{0pt}%
   \setlength{\partopsep}{0pt}%
   \tabbing}
  {\endtabbing} 



% \theoremstyle{definition}
% \newtheorem{def}[thm]{Definition} % definition numbers are dependent on theorem numbers



%%% Local Variables:
%%% mode: latex
%%% TeX-master: "main"
%%% End:

\usepackage{eucal}
\usepackage{url}
\usepackage{tikz}
\usepackage{amsfonts,amsmath}
\begin{document}

\title{Verifying Snapping Mechanism - Fixed Point Implementation Version}
\author{Jiawen Liu}

\date{\today}

\maketitle
In order to verify an implementation of the snapping mechanism
\cite{mironov2012significance} differentially private, we will extend
the logic proposed in~\cite{barthe2016proving} and we will take
inspiration from the fixed
point error semantics from
\cite{Ramananandro2016unified,Martel2006higher,Becker2018verified,Moscato2017Automatic}.

\section{Preliminary Definitions}
\begin{defn}
[Laplace mechanism \cite{dwork2006calibrating}]
Let $\epsilon > 0$. The Laplace mechanism  $\lapmech_{\epsilon}$: $\real \to \distr(\real)$ is defined by $\lapmech(t) = t + v$, where $v \in \real$ is drawn from the Laplace distribution $\laplace(\frac{1}{\epsilon})$.
\end{defn}
%
%
%

\section{Syntax}
\[\begin{array}{llll}
\mbox{Programs} & \prog & ::= &
	%
    \varx = \expr ~|~ \varx \samplel \edistr
	%
	~|~ \prog ; \prog  
	\\
%
\mbox{Expr} & \expr & ::= & \rval
	%
	~|~ \varx  ~|~ \expr \bop \expr
	%
	~|~ \ln (\expr) ~|~ - \expr ~|~ \clamp_{\rval}(\expr)
	%
	~|~ \round{\expr}_{\rval}  \\
%
\mbox{Binary Operation} & \bop & ::= & + ~|~ - ~|~ \times ~|~ \div \\
%
%
%
\mbox{Value} & \valv & ::= & \fval_{(\rval, \rval)}\\
%
\mbox{Distr} & \edistr & ::= & \uniform(0, 1]
%
	~|~ \uniform\{-1, 1\}\\ 
%
\mbox{Error} & \err & ::= & (\rval, \rval) \\
%
\mbox{Env} & \trsmem & ::= & \cdot ~|~ \trsmem[x \mapsto (\fval_{(\rval, \rval)})]\\
%
\mbox{Type} & \tau & ::= & \float ~|~ \real ~|~ \float_{\real \times \real}
\end{array}
\]
%
We consider a simple imperative language $\prog$ with assignments, random sampling and sequencing, and simple arithmetic expressions $\expr$ with real number $\rval$, fixed point number $\fval$, variable $\varx$, binary operations and unary operations.
The errors $\err$ are represented by a pair of real number. The environment maps variable to a fixed point number and an error.
%
%
%
\section{Operational Semantics}
%
The semantics for expressions with relative
fixed point computation error are shown in
Figure~\ref{fig_semantics_prog}. 
$\floaten (x)$ represents round the number to its nearset $ulp$ when the number overflowed.
%
In all the rules, $\eta$ is the scale of the fixed point arithmetic.
%
%
\begin{figure}
\boxed{\trsmem, \expr \trsto \fval_{(\rval, \rval)}
: Env \times Expr \trsto Value}
\begin{mathpar}
\inferrule*[right = var]
{
	\trsmem(\varx) 
	= (\fval_{( \ubar{\rval}, \bar{\rval} )})
}
{
	\trsmem, \varx
	\trsto
	(\fval_{( \ubar{\rval}, \bar{\rval} )})
}
\and
%
\inferrule*[right = val]
{
	\fval = \floaten(\rval)
	~~
	\fval \neq \rval
	\and
	\rval \geq 0
}
{
	\trsmem, \rval
	\trsto
	\big(\fval_{(\frac{\rval}{(1 + \eta)}, \rval(1 + \eta))} 
	\big)
}
%
\and
%
\inferrule*[right = val-eq]
{
	\fval = \floaten(\rval)
	~~~~
	\fval = \rval
}
{
	\trsmem, \rval
	\trsto
	(\fval_{(\rval, \rval)} )
}
%
\and
%
\extend{
\inferrule*[right = bop]
{
	\trsmem, \expr_1 \trsto 
	(\fval_{1(l_1, r_1)})
	~~~~
	\trsmem, \expr_2 \trsto (\fval_{2(l_2, r_2)})
	~~~~
	\fval = \floaten(\fval_1 \bop \fval_2)
	\and
	\bop \in \{+, - \} 
	\lor 
	\big(
	\bop \in \{\times , \div \} \land
	 \fval_1 \geq 0 \land \fval_2 \geq 0
	 \big)
}
{
    \trsmem, \expr_1 \bop \expr_2
    \trsto
    \fval_{
    \big(
        (l_1 \bop l_2) - \eta, 
        (r_1 \bop r_2 ) + \eta
        \big)}
}
}
%
\and
%
\extend{
\inferrule*[right = bop-n]
{
	\trsmem, \expr_1 \trsto 
	(\fval_{1(l_1, r_1)})
	~~~~
	\trsmem, \expr_2 \trsto (\fval_{2(l_2, r_2)})
	~~~~
	\fval = \floaten(\fval_1 \bop \fval_2)
	\and
	\bop \in \{\times, \div\}
	\and
	\fval_1 < 0 \lor \fval_2 < 0
}
{
    \trsmem, \expr_1 \bop \expr_2
    \trsto
    \fval_{\big(
        \min{(\{l,r\}_1 \bop \{l,r\}_2)} - \eta, 
        \max{(\{l,r\}_1 \bop \{l,r\}_2)} + \eta
        \big)}
}
}
%
\and
%
\extend{
\inferrule*[right = uop]
{
	\trsmem, \expr \trsto 
	(\fval_{1(l, r)})
	~~~~
	\fval = \floaten(\uop (\fval_1))
	\and 
	\uop \in \{\ln, \clamp, \round{}\} 
}
{
    \trsmem, \uop(\expr)
    \trsto
    \Big(\fval_
    {\big(
     \uop(l) - \eta, 
        (\uop(r) + \eta)
        \big)}
    \Big)
}
}
%
\and
%
\extend{
\inferrule*[right = neg]
{
	\trsmem, \expr \trsto 
	(\fval_{1(l, r)})
	~~~~
	\fval = \floaten(\uop (\fval_1)) 
}
{
    \trsmem, \uop(\expr)
    \trsto
    \Big(\fval_
    {\big(
     -r - \eta, 
        -l + \eta)
        \big)}
    \Big)
}
}
%
\end{mathpar}
\caption{Semantics of Transition for Expressions with Absolute Fixed Point Error}
\label{fig:op_semantics}
\end{figure}
%
%
%
\begin{figure}
\boxed{\trsenv, \prog \trsto \trsenv}
\begin{mathpar}
\inferrule*[right = asg]
{
	\trsenv, \expr \trsto (\fval, \err )
}
{
	\trsenv, \varx = \expr
	\trsto
	\trsenv[\varx \mapsto (\fval, \err )]
}
%
~~
%
\inferrule*[right = consq]
{
	\trsenv, \prog_1  \trsto \trsenv_1
	\and
	\trsenv_1, \prog_2  \trsto \trsenv_2
}
{
	\trsenv, \prog_1; \prog_2
	\trsto
	\trsenv_2
}
%
~~
%
\inferrule*[right = sample]
{
	 \fval \leftarrow \edistr^{\diamond}
	 % \and
	 % \rval_1 \leq \fval \leq \rval_2
	 ~~~~
	 \rval = \fval
}
{
	\trsenv, \varx \samplel \edistr
	\trsto
	\trsenv[\varx \mapsto (\fval, (\rval, \rval))]
}
\end{mathpar}
\caption{Semantics of Transition with Relative Fixed Point Error Propagation for Programs}
\label{fig_semantics_prog}
\end{figure}


\begin{figure}
\boxed{\expr \fbigstep \fval }
\begin{mathpar}
\inferrule*[right = rval]
{
	\floaten(\rval) = \fval
}
{
	\rval
	\fbigstep
	\fval
}
%
~~
%
\inferrule*[right = fval]
{
	\empty
}
{
	\fval
	\fbigstep
	\fval
}
%
~~
%
\inferrule*[right = fbop]
{
	\expr^1 \fbigstep \fval_1
	~~~
	\expr^2 \fbigstep \fval_2
	~~~
	\floaten(\fval_1 \bop \fval_2) = \fval
}
{
    \expr^1 \bop \expr^2 \fbigstep \fval
}
%
~~
%
\inferrule*[right = fuop]
{
	\expr \fbigstep \fval_1,
	~~~
	\floaten(\uop (\fval_1)) = \fval
}
{
    \uop(\expr) \fbigstep \fval
}
\end{mathpar}
\caption{Semantics of Evaluation in Fixed Point Computation}
\label{fig_imp_real_semantics_exp}
\end{figure}

\begin{figure}
\boxed{\expr \rbigstep \rval }
\begin{mathpar}
\inferrule*[right = rval]
{
	\empty
}
{
	\rval
	\rbigstep
	\rval
}
%
~~
%
\inferrule*[right = rval]
{
	\empty
}
{
	\fval
	\rbigstep
	\fval
}
%
~~
%
\inferrule*[right = rbop]
{
	\expr^1 \rbigstep \rval_1
	~~~
	\expr^2 \rbigstep \rval_2
	~~~
	\rval_1 \bop \rval_2 = \rval
}
{
    \expr^1 \bop \expr^2 \rbigstep \rval
}
%
~~
%
\inferrule*[right = ruop]
{
	\expr \rbigstep \rval_1,
	~~~
	\uop (\rval_1) = \rval
}
{
    \uop(\expr) \rbigstep \rval
}
\end{mathpar}
\caption{Semantics of Evaluation in Real Computation}
\label{fig_real_semantics_exp}
\end{figure}

\clearpage
% \mg{I suggest to split the theorem proof by adding a lemma that states
% what you want about expressions. This will be used in the case of
% assignment in the theorem.}
\begin{defn}[Bounded Environment]
An environment $\trsenv$ is bounded if and only if that:
$\forall x \in dom(\trsenv)$ s.t. $\trsenv(x) =(\fval, (\ubar{\rval}, \bar{\rval}))$,
 then 
$\ubar{\rval} \leq \fval \leq \bar{\rval}$) 
\end{defn}
%
%
\begin{thm}[Expression Soundness]
\label{thm:expsound}
For any $\expr$, if there exists a transition 
$\trsenv, \expr \trsto (\fval, (\ubar{\rval}, \bar{\rval}))$ 
and $\trsenv$ is a bounded transaction environment
%
then:
%
$$\ubar{\rval} \leq \fval \leq \bar{\rval}.$$
%
\end{thm}
%
%
%
\begin{thm}[Program Soundness]
\label{thm:progsound}
For any $\prog$, if there exists a transition 
$\trsenv, \prog \trsto \trsenv'$ and $\trsenv$ is a bounded transaction environment 
then 
$\trsenv'$ is also a bounded environment.
\end{thm}
%
\begin{proof} of \textbf{Theorem \ref{thm:progsound}}.
%
\\
%
Induction on transition rule of $\prog$, by assumption, we know $\trsenv$ is a bounded environment $(\star)$.
\begin{itemize}
	\caseL{
	\[
	\inferrule*[right = consq]
	{
	\trsenv, \prog_1  \trsto \trsenv_1
	\and
	\trsenv_1, \prog_2  \trsto \trsenv_2
	}
	{
	\trsenv, \prog_1; \prog_2
	\trsto
	\trsenv_2
	}
	\]
	}
	We need to show $\trsenv_2$ is a bounded environment.\\
	%
	Since we know $\trsenv$ is a bounded environment by assumption $(\star)$, by induction hypothesis, we have:
	%
	\\
	$\trsenv_1$ and $\trsenv_2$ are all bounded environment. This case is proved.
%
	\caseL{\[
	\inferrule*[right = sample]
	{
		 \fval \leftarrow \edistr^{\diamond}
		 \and
	 	\rval_1 = \fval = \rval_2 ~ (\square)
	}
	{
		\trsenv, \varx \samplel \edistr
		\trsto
		\trsenv[\varx \mapsto (\fval, (\rval_1, \rval_2))]
	}
	\]}
	We need to show $\trsenv[\varx \mapsto (\fval, (\rval_1, \rval_2))]$ is a bounded environment.\\
	%
	Since we know $\trsenv$ is a bounded environment and $\rval_1 = \fval =\rval_2$ by assumption $(\star)$ and $(\square)$ .
	%
	\\
	%
	So we know $\trsenv[\varx \mapsto (\fval, (\rval_1, \rval_2))]$ is also a bounded environment.
%
	\caseL{\[
	\inferrule*[right = asg]
	{
		\trsenv, \expr \trsto (\fval, (\ubar{\rval}, \bar{\rval}) )
	}
	{
		\trsenv, \varx = \expr
		\trsto
		\trsenv[\varx \mapsto (\fval, (\ubar{\rval}, \bar{\rval}) )]
	}
	\]}
	We need to show: $\trsenv[\varx \mapsto (\expr, \err )]$ is a bounded environment.\\
	%
	By assumption $(\star)$ we know: $\trsenv$ is already a bounded environment. It is sufficient to show:\\
	%
	$\ubar{\rval} \leq \fval \leq \bar{\rval}$.\\
	%
	This is proved by the \textbf{Theorem \ref{thm:expsound}} (expression soundness).
\end{itemize}
\end{proof}
%
%
\begin{proof} of \textbf{Theorem \ref{thm:expsound}}.
%
\\
%
Induction on the transition rule of $\expr$, by assumption, we know $\trsenv$ is a bounded environment $(\star)$.
%
	\begin{itemize}
%
	\caseL{
	\[\inferrule*[right = var]
		{
			\trsenv(\varx) 
			= (\fval, ( \ubar{\rval}, \bar{\rval} ))
		}
		{
			\trsenv, \varx
			\trsto
			(\fval, ( \ubar{\rval}, \bar{\rval} ))
		}\]
		}
	By the assumption, we have $\forall \varx \in dom(\trsenv)$ s.t. $\trsenv(\varx) = (\fval, (\ubar{\rval}, \bar{\rval}))$, 
	$\ubar{\rval} \leq \fval \leq \bar{\rval}$.
	This case is proved.
	%
	%
	\caseL{
	\[\inferrule*[right = val]
		{
			\fval = \floaten(\rval)
			\and
			\fval \neq \rval
			\and
			\rval \geq 0
		}
		{
			\trsenv, \rval
			\trsto
			\big(\fval, 
			(\frac{\rval}{(1 + \eta)}, \rval(1 + \eta)) 
			\big)
		}
		\]
	}
	%
	By the definition of fixed point rounding error and $\rval \geq 0$, we have:
	%
	$\frac{\rval}{(1 + \eta)}
	\leq \fval \leq
	\rval(1 + \eta)$
	%
	%
	%
	\caseL{\[
		\inferrule*[right = val-eq]
		{
			\fval = \floaten(\rval)
			~~~~
			\fval = \rval
		}
		{
			\trsenv, \rval
			\trsto
			(\fval, (\rval, \rval) )
		}
	\]}
	%
	It is trivial that $\rval \leq \fval = \floaten(\rval) = \rval \leq \rval$
	%
	\caseL{\[
\extend{
	\inferrule*[right = bop]
{
	\trsmem, \expr_1 \trsto 
	(\fval_{1(\ubar{\rval_1}, \bar{\rval_1})})
	~~~~
	\trsmem, \expr_2 \trsto (\fval_{2(\ubar{\rval_2}, \bar{\rval_2})})
	~~~~
	\fval = \floaten(\fval_1 * \fval_2)
}
{
    \trsmem, \expr_1 * \expr_2
    \trsto
    \big(
    (\ubar{\rval_1} * \ubar{\rval_2}) - \eta, 
    (\bar{\rval_1} * \bar{\rval_2} ) + \eta
    \big)
}
}
	\]}
	%
	It is needed to show $\frac{\ubar{\rval_1} * \ubar{\rval_2}}{(1 + \eta)}
	\leq \fval \leq 
	(\bar{\rval_1} * \bar{\rval_2})(1 + \eta)$.\\
	%
	By induction hypothesis on $(\diamond)$ and $(\triangle)$, we have:\\
	%
	(1) $\ubar{\rval_1} \leq \fval_1 \leq \bar{\rval_1}$ holds. 
	%
	(2) $\ubar{\rval_2} \leq \fval_2 \leq \bar{\rval_2}$ holds.\\
	%
	Let $\bar{\rval{'}} = 
	\bar{\rval_1} * \bar{\rval_2}$ and 
	%
	$\ubar{\rval{'}} = \ubar{\rval_1} * \ubar{\rval_2}$.
	%
	\\
	%
	By (1), (2), we have:
	%
	$\fval \geq 0
	\land
	\ubar{\rval{'}}
	\leq \fval_2 \bop \fval_1
	\leq \bar{\rval{'}}$.\\
	%
	Then by absolute error of fixed point rounding, we have:\\
	%
	$\ubar{\rval{'}} -  \eta
	\leq \floaten(\fval_2 \bop \fval_1) = \fval
	\leq (\bar{\rval{'}
	}  + \eta)$.
	%
	%
    %
	\caseL
	{
	\[
	%
\extend{
	\inferrule*[right = uop]
	{
		\trsmem, \expr \trsto 
		(\fval_{1(\ubar{\rval}, \bar{\rval})})
		~~~~
		\fval = \floaten(\uop (\fval_1)) 
	}
	{
	    \trsmem, \uop(\expr)
	    \trsto
	    \Big(\fval_
	    {\big(
	     \uop(\ubar{\rval}) - \eta, 
	        (\uop(\bar{\rval}) + \eta)
	        \big)}
	    \Big)
	}
}
	\]
	}
	%
 	It is sufficient to show 
	$\frac{\uop(\ubar{\rval})}{1 + \eta}
	\leq \fval \leq 
	(\uop(\bar{\rval}))(1 + \eta)$.\\
	%
	By induction hypothesis on $(\diamond)$, we have:\\
	%
	(1) $\ubar{\rval} \leq \fval{'} \leq \bar{\rval}$ holds.
	%
	\\
	%
	By (1) and monotone of unary operators, we have:
	%
	$\uop(\ubar{\rval})
	\leq \uop(\fval{'})
	\leq \uop(\bar{\rval})$.\\
	%
	Then by absolute error of fixed point rounding, we have:\\
	%
	$\uop(\ubar{\rval}) -  \eta
	\leq \floaten(\uop(\fval{'})) = \fval
	\leq \uop(\bar{\rval} + \eta)$.
	%
	This case is proved.
	%
	%
\end{itemize}
%
\end{proof}

\newpage
\section{Snapping Mechanism}

\begin{defn}
[$\snap(a) : A \to \distr(\real)$]
Given privacy parameter $\epsilon$, the Snapping mechanism $\snap(a)$ is defined as:
\[
	U \samplel \uniform(0,1); S \samplel \uniform\{-1, 1\};
	x = \clamp_B \big(
	\round{f(a) + \frac{1}{\epsilon} \times S \times \ln (U)}_{\Lambda}
	\big)
\]
where $f(a)$ represents a value that the query function $f$ be evaluted over input database $a \in A$, $\epsilon$ is the privacy parameter, $B$ is the clamping argument and $\Lambda$ is the rounding argument satisfying $\lambda = 2^k$ where $2^k$ is the smallest power of 2 greater or equal to the $\frac{1}{\epsilon}$.
%
% \\
% %
% Given $U = u, S = s$, let $\expr_{\snap'} = f(a) + s \times \ln (u) \div \epsilon$,
% $\expr_{\snap''} = \clamp_B \big( \round{\rval_{\snap'} }_{\Lambda} \big)$
% %
% \\
% %
% Let $\snap'(a)$ be the same as $\snap(a)$ given $U = u, S = s$ without rounding and clamping steps, i.e., $\snap'(a): y = \expr_{\snap'}$,
% and $\snap''(a): z = \expr_{\snap''}$.
\end{defn}
%
%
%
\begin{lem}[sign]
\label{lem:sign}
For any input $a$ some $s$ and output $\valv$ s.t.
$\clamp_B \big(
	\round{f(a) + \frac{1}{\epsilon} \times s \times \ln (u)}_{\Lambda}
	\big)
	\fbigstep \valv$ 
or $\clamp_B \big(
	\round{f(a) + \frac{1}{\epsilon} \times s \times \ln (u)}_{\Lambda}
	\big)
	\rbigstep \valv$,
%
\begin{enumerate}  
	\item if $\valv < f(a)$, then $s = 1$;
	\item if $\valv = f(a)$, then $s = 1$ or $ -1$;
	\item if $\valv > f(a)$, then $s = -1$.
\end{enumerate}
\end{lem}
%
%
%
\begin{lem}
[clampL]
\label{lem:clampl}
For any input $a$:
\begin{enumerate}
\item 
if 
$\clamp_B \big(
	\round{f(a) + \frac{1}{\epsilon} \times s \times \ln (u)}_{\Lambda}
	\big)
	\fbigstep -B$ or $B$, then $u \in (0, \fvalR)$ for some $\fvalR$.
	%
\item
if
$\clamp_B \big(
	\round{f(a) + \frac{1}{\epsilon} \times s \times \ln (u)}_{\Lambda}
	\big)
	\rbigstep -B$ or $B$, then $u \in (0, \rvalR)$ for some $\rvalR$.
\end{enumerate}
\end{lem}


\begin{lem}[clampR]
\label{lem:clampr}
Let $b$ be the largest number rounded by $\Lambda$ that is smaller than $B$, for any input $a$, and some $s$:
\begin{enumerate}
\item 
if 
$\clamp_B \big(
	\round{f(a) + \frac{1}{\epsilon} \times s \times \ln (u)}_{\Lambda}
	\big)
	\fbigstep -B$ and $u \in (\fvalL, \fvalR)$ for some $\fvalL, \fvalR$.
	%
	%
	 then:
	 %
	 \\
	 %
	$f(a) + \frac{1}{\epsilon} \times s \times \ln (\fvalR) \fbigstep -b - \frac{\Lambda}{2}$.
%
\item 
if 
$\clamp_B \big(
	\round{f(a) + \frac{1}{\epsilon} \times s \times \ln (u)}_{\Lambda}
	\big)
	\fbigstep B$ and $u \in (\fvalL, \fvalR)$ for some $\fvalL, \fvalR$.
	%
	%
	 then:
	 %
	 \\
	 %
	$f(a) + \frac{1}{\epsilon} \times s \times \ln (\fvalR) \fbigstep b + \frac{\Lambda}{2}$.
%
\item
	%
	%
if
$\clamp_B \big(
	\round{f(a) + \frac{1}{\epsilon} \times s \times \ln (u)}_{\Lambda}
	\big)
	\rbigstep -B$ and $u \in (\rvalL, \rvalR)$ for some $\rvalL, \rvalR$, then:
	%
	 %
	 \\
	 %
	$f(a) + \frac{1}{\epsilon} \times s \times \ln (\rvalR) \rbigstep -b - \frac{\Lambda}{2}$.
%
\item
	%
	%
if
$\clamp_B \big(
	\round{f(a) + \frac{1}{\epsilon} \times s \times \ln (u)}_{\Lambda}
	\big)
	\rbigstep B$ and $u \in (\rvalL, \rvalR)$ for some $\rvalL, \rvalR$, then:
	 %
	 \\
	 %
	$f(a) + \frac{1}{\epsilon} \times s \times \ln (\rvalR) \rbigstep b + \frac{\Lambda}{2}$.
%
\end{enumerate}
\end{lem}
%
%
\begin{lem}[clampId]
\label{lem:clampid}
For any input $a$:
\begin{enumerate}
\item 
if 
$\clamp_B \big(
	\round{f(a) + \frac{1}{\epsilon} \times s \times \ln (u)}_{\Lambda}
	\big)
	\fbigstep \valv$ and $\valv \in (-B, B)$, then:
	%
	$\round{f(a) + \frac{1}{\epsilon} \times s \times \ln (u)}_{\Lambda}
	\fbigstep \valv$
%
\item
	%
	%
if
$\clamp_B \big(
	\round{f(a) + \frac{1}{\epsilon} \times s \times \ln (U)}_{\Lambda}
	\big)
	\rbigstep \valv$ and $\valv \in (-B, B)$, then:
	%
	$\round{f(a) + \frac{1}{\epsilon} \times s \times \ln (u)}_{\Lambda}
	\rbigstep \valv$
\end{enumerate}
\end{lem}


\begin{lem}[roundRL]
\label{lem:roundrl}
For any input $a$ and output $\valv \in (-B, B)$,
let $(\fvalL, \fvalR)$ be the largest range s.t. 
$\forall u \in (\fvalL, \fvalR)$ and some $s$:
%
$$
\clamp_B \big(
	\round{f(a) + \frac{1}{\epsilon} \times s \times \ln (u)}_{\Lambda}
	\big)
	\fbigstep \valv,$$
then:
$$
f(a) + \frac{1}{\epsilon} \times s \times 
\ln (\fvalL)
\fbigstep \valv - \frac{\Lambda}{2}
~~~~~~~~~~~~~~~~
f(a) + \frac{1}{\epsilon} \times s \times 
\ln (\fvalR)
\fbigstep \valv + \frac{\Lambda}{2}
$$
\end{lem}
%%
%%
%%
\begin{lem}[roundId]
\label{lem:roundid}
if 
$\clamp_B \big(
	\round{f(a) + \frac{1}{\epsilon} \times s \times \ln (u)}_{\Lambda}
	\big)
	\fbigstep \round{f(a)}_{\Lambda}$, then $u \in (\fvalL, 1]$ for some $\fvalL$.
	%
\item
if
$\clamp_B \big(
	\round{f(a) + \frac{1}{\epsilon} \times s \times \ln (u)}_{\Lambda}
	\big)
	\rbigstep \round{f(a)}_{\Lambda}$, then $u \in (\rvalL, 1]$ for some $\rvalL$.
\end{lem}
%
%%%
%
%
%
\begin{lem}[idealRL]
\label{lem:idealrl}
For any input $a$ and output $\valv \in (-B, B)$,
let $(\rvalL, \rvalR)$ be the largest range where 
$\forall u \in (\rvalL, \rvalR)$ and some $s$ s.t.:
%
$\clamp_B \big(
	\round{f(a) + \frac{1}{\epsilon} \times s \times \ln (u)}_{\Lambda}
	\big)
\rbigstep \valv
$, then:
%
$$
\rvalL = e^{\epsilon(\valv - \Lambda/2 - f(a))/s}
~~~~~~~~~~~~~~~~~~~~~~~~~~
\rvalR = e^{\epsilon(\valv + \Lambda/2 - f(a))/s}
$$
and 
$\Pr[\snap(a) = \valv] = 1/2(\rvalR - \rvalL)$.
\end{lem}

\begin{lem}[idealDP]
\label{lem:idealdp}
The $\snap$ mechanism in real computation is differentially private.
\end{lem}

\newpage
\section{Main Theorem}

\begin{thm}
[The $\snap$ mechanism is $(6.9( \epsilon + 1) \eta + \epsilon, 0)-$differentially private]
%
Consider the $\snap$ mechanism defined as before. For any privacy parameter $\epsilon$ representable in fixed point computation, for all possible outputs $\valv$ in fixed point computation and pairs of adjacent input $a$ and $a'$ satisfying $|f(a) - f(a')| \leq 1$, the privacy loss between $\snap(a)$ and $\snap(a')$ is bounded by $e^{\epsilon + 23 B \epsilon \eta}$.
\end{thm}

\begin{proof}
%
%
We consider following setting:
\\
1. $\valv$ is a possible output of $\snap(a)$ in fixed point computation,
%
\\
2. any pair of adjacent database $a'$ and $a$ where $|f(a) - f(a')| \leq 1$
%
\\
3. and any parameter $\epsilon$ representable in fixed point computation.
%
\\
Without loss of generality, we assume $f(a) + 1 = f(a') ~ (\diamond)$.
%
\\
%
%
Let $(\rvalL, \rvalR)$, $(\rvalL{'}, \rvalR{'})$ be the range where 
$\forall u \in (\rvalL, \rvalR)$ and some $s$, 
$\forall u' \in (\rvalL{'}, \rvalR{'})$ and some $s'$, s.t.:
%
$$\clamp_B \big(
	\round{f(a) + \frac{1}{\epsilon} \times s \times \ln (u)}_{\Lambda}
	\big)
	 \rbigstep \valv; 
~~~~~~~~~~~~~~~~~~
\clamp_B \big(
	\round{f(a') + \frac{1}{\epsilon} \times s' \times \ln (u')}_{\Lambda}
	\big)
	 \rbigstep \valv.$$
%
By \textbf{Lemma \ref{lem:idealrl}}, we have:
$\Pr[\snap(a) = \valv] = \frac{1}{2} (\rvalR - \rvalL)$ 
and $\Pr[\snap(a') = \valv] = \frac{1}{2} (\rvalR{'} - \rvalL{'})$.
%
\\
%
By \textbf{Lemma \ref{lem:idealdp}}, we can get:
\[
	e^{-\epsilon} \leq \frac{\Pr[\snap(a)]}{\Pr[\snap(a')]}
	= \frac{\rvalR - \rvalL}{\rvalR{'} - \rvalL{'}} \leq e^{\epsilon}
\]
%
\\
%
Let $(\fvalL, \fvalR)$, $(\fvalL{'}, \fvalR{'})$ be the largest range where 
$\forall u \in (\fvalL, \fvalR)$ and some $s$, 
$\forall u' \in (\fvalL{'}, \fvalR{'})$ and some $s'$, s.t.:
%
$$
\clamp_B \big(
	\round{f(a) + \frac{1}{\epsilon} \times s \times \ln (u)}_{\Lambda}
	\big)
	\fbigstep \valv; 
~~~~~~~~~~~~~~~~~~
\clamp_B \big(
	\round{f(a') + \frac{1}{\epsilon} \times s' \times \ln (u')}_{\Lambda}
	\big)
	\fbigstep \valv.$$
%
To show the privacy loss between $\snap(a)$ and $\snap(a')$ is bounded by $e^{\epsilon + 23 B \epsilon \eta}$, it’s sufficient to show:
%
%
$h(|{\rvalR} - {\rvalL}|) \leq |\fvalR - \fvalL| \leq g(|{\rvalR} - {\rvalL}|)$ and 
$h(|{\rvalR{'}} - {\rvalL{'}}|) \leq |\fvalR{'} - \fvalL{'}|
\leq g(|{\rvalR{'}} - {\rvalL{'}}|) $,
%
s.t.:
%
\[
\extend{
	-6.9( \epsilon + 1) \eta - \epsilon
	\leq \ln\left(
	 \frac{h(|{\rvalR} - {\rvalL}|)}{g(|{\rvalR{'}} - {\rvalL{'}}|)} 
	 \right) 
	\leq
	\ln\left( 
	\frac{g(|{\rvalR} - {\rvalL}|)}{h(|{\rvalR{'}} - {\rvalL{'}}|)} 
	\right)
	\leq 6.9( \epsilon + 1) \eta + \epsilon.
	}
\]
		%
		%
		%
		%
		%
		% \caseL{
		% $\boldsymbol{\valv \in (-B, \round{f(a)}_{\Lambda})} ~ (\star) $
		% }
		% %
		% \subcaseL{
		% $\boldsymbol{
		% \round{f(a)}_{\Lambda} \leq 0 
		% %
		% \lor \bigg( \round{f(a)}_{\Lambda} > 0 \land \valv \in (-B, 0) \bigg) } ~ (\star_1) $
		% }
		%
		Let $\valv_1 = \valv - (\frac{\Lambda}{2})$,
		$\valv_2 = \valv + (\frac{\Lambda}{2})$.
		%
		\\
		%
		By \textbf{Lemma \ref{lem:idealrl}} and $(\star) $, we know: $\rvalL = e^{\epsilon(\valv_1 - f(a))}$ and 
		$\rvalR = e^{\epsilon(\valv_2 - f(a))}$.
		%
		\\
		%
		By \textbf{Lemma \ref{lem:roundrl}}, we know:
		%
		$$f(a) + \frac{1}{\epsilon} \times \ln(\fvalL) \fbigstep \valv_1 ~~ (1)
		~~~~~~~~~~~~~~~~~
		f(a) + \frac{1}{\epsilon} \times \ln(\fvalR) \fbigstep \valv_2 ~~ (2).$$
		%
	In order to bound $\fvalL$ in terms of $\rvalL$, we have the following derivation where $P$ be the real representation of $\fvalL ~~ (\diamond)$:
		%
{\scriptsize
\begin{mathpar}
	\inferrule*[right=bop]
	{
	\inferrule*[right=bop]
	{
		\inferrule*[right = uop]
		{
			\inferrule*[right = val-eq]
			{
				P ~~~ \text{representable}
			}
			{
				\trsenv, P 
				\trsto
				(\fvalL,
				(P, P))
			}
		}
		{
				\trsenv,
				\ln(P) 
				\trsto
				(\ln(\fvalL), \ln(P) - \eta,
				\ln(P) + \eta)
			}
			\and
			\inferrule*[right = val-eq]
			{
			\frac{1}{\epsilon}~ \text{representable}
			}
			{
				\frac{1}{\epsilon} \trsto
				(\frac{1}{\epsilon},
				(\frac{1}{\epsilon},
			\frac{1}{\epsilon}))	
			}
		}
			{
					\trsenv,
					\frac{1}{\epsilon} \times \ln(P) 
					\trsto
				\extend{
				(\frac{1}{\epsilon} \times \ln(\fvalL),
				(\frac{1}{\epsilon} \times (\ln(P) - \eta) - \eta,
					%
				\frac{1}{\epsilon} \times (\ln(P) + \eta) + \eta)
				}
				}
				\and
				\inferrule*[right=val-eq]
				{f(a) ~~ \text{representable}}
				{	[],
				f(a) \trsto
				(f(a),(f(a),f(a)))
				}
			}
				{
				\trsenv,
				f(a) + \frac{1}{\epsilon} \times \ln(P)
				\trsto
				\extend{
				\bigg(
				f(a) + \frac{1}{\epsilon} \times \ln(\fvalL),
						%
				\big( (f(a) + 
				(\frac{1}{\epsilon} \times (\ln(P) - \eta ) - 2 \eta)
				,
				%
				(f(a) + 
				(\frac{1}{\epsilon} \times (\ln(P) + \eta ) + 2 \eta)
				 \big)}
				\bigg)
				}
		\end{mathpar}	
}	
%
%
\extend{
		By soundness theorem and (1), we have:
		%
		\[
		(f(a) + (\frac{1}{\epsilon} \times (\ln(P) - \eta ) - 2 \eta)
		\leq \valv_1 \leq
		(f(a) + 
		(\frac{1}{\epsilon} \times (\ln(P) + \eta ) + 2 \eta)
		\]
		%
		%
		Then we have:
		$$
		e^{\epsilon 
		 (\valv_1 - f(a) - 2\eta) - \eta}
		\leq P \leq
		e^{\epsilon 
		(\valv_1 - f(a) + 2\eta) + \eta}.
		$$
		%
		%
		By $(\diamond)$, we have:
		$$
		e^{\epsilon 
		 (\valv_1 - f(a) - 2\eta) - \eta}
		\leq \fvalL \leq
		e^{\epsilon 
		(\valv_1 - f(a) + 2\eta) + \eta}.
		.$$
}		%
		%
		%
	In order to bound $\fvalR$ in terms of $\rvalR$, we have the following derivation where $P$ be the real representation of $\fvalR ~~ (\diamond)$:
		%
{\scriptsize
\begin{mathpar}
	\inferrule*[right=bop]
	{
	\inferrule*[right=bop]
	{
		\inferrule*[right = uop]
		{
			\inferrule*[right = val-eq]
			{
				P ~~~ \text{representable}
			}
			{
				\trsenv, P 
				\trsto
				(\fvalR,
				(P, P))
			}
		}
		{
				\trsenv,
				\ln(P) 
				\trsto
				(\ln(\fvalR), \ln(P) - \eta,
				\ln(P) + \eta)
			}
			\and
			\inferrule*[right = val-eq]
			{
			\frac{1}{\epsilon}~ \text{representable}
			}
			{
				\frac{1}{\epsilon} \trsto
				(\frac{1}{\epsilon},
				(\frac{1}{\epsilon},
			\frac{1}{\epsilon}))	
			}
		}
			{
					\trsenv,
					\frac{1}{\epsilon} \times \ln(P) 
					\trsto
				\extend{
				(\frac{1}{\epsilon} \times \ln(\fvalR),
				(\frac{1}{\epsilon} \times (\ln(P) - \eta) - \eta,
					%
				\frac{1}{\epsilon} \times (\ln(P) + \eta) + \eta)
				}
				}
				\and
				\inferrule*[right=val-eq]
				{f(a) ~~ \text{representable}}
				{	[],
				f(a) \trsto
				(f(a),(f(a),f(a)))
				}
			}
				{
				\trsenv,
				f(a) + \frac{1}{\epsilon} \times \ln(P)
				\trsto
				\extend{
				\bigg(
				f(a) + \frac{1}{\epsilon} \times \ln(\fvalR),
						%
				\big( (f(a) + 
				(\frac{1}{\epsilon} \times (\ln(P) - \eta ) - 2 \eta)
				,
				%
				(f(a) + 
				(\frac{1}{\epsilon} \times (\ln(P) + \eta ) + 2 \eta)
				 \big)}
				\bigg)
				}
		\end{mathpar}	
}
%
%
\extend{
		By soundness theorem and (2), we have:
		\[
		(f(a) + (\frac{1}{\epsilon} \times (\ln(P) - \eta ) - 2 \eta)
		\leq \valv_2 \leq
		(f(a) + 
		(\frac{1}{\epsilon} \times (\ln(P) + \eta ) + 2 \eta)
		\]
		%
		%
		Then we have:
		$$
		e^{\epsilon 
		 (\valv_2 - f(a) - 2\eta) - \eta}
		\leq P \leq
		e^{\epsilon 
		(\valv_2 - f(a) + 2\eta) + \eta}.
$$
		%
		%
		By $(\diamond)$, we have:
		$$
		e^{\epsilon 
		 (\valv_2 - f(a) - 2\eta) - \eta}
		\leq \fvalR \leq
		e^{\epsilon 
		(\valv_2 - f(a) + 2\eta) + \eta}.
		.$$	
%
		In the same way, we have the bounds for $\fvarL{'}, \fvarR{'}$ for input $a'$ as:
%
		$$e^{\epsilon 
		 (\valv_1 - f(a') - 2\eta) - \eta}
		\leq \fvarL{'} \leq
		e^{\epsilon 
		(\valv_1 - f(a') + 2\eta) + \eta}.$$
		%
		%
		$$ 
		e^{\epsilon 
		 (\valv_2 - f(a') - 2\eta) - \eta}
		\leq \fvarR{'} \leq
		e^{\epsilon 
		(\valv_2 - f(a') + 2\eta) + \eta}$$
%
		Let  
		$h{(\fvalL) } = e^{\epsilon 
		 (\valv_1 - f(a) - 2\eta) - \eta}$ 
		 be the lower bound for $\fvalL$
		and $g(\fvalR) = e^{\epsilon 
		(\valv_2 - f(a) + 2\eta) + \eta}$ 
		be the upper bound for $\fvalR$
		%
		We can have:
		\[
		\begin{array}{ll}
		\frac{g(\fvalR)}{\rvarR} 
		& = e^{\epsilon 
		(\valv_2 - f(a) + 2\eta) + \eta}/e^{\epsilon(\valv_2 - f(a))}
		 =
		e^{ 
		(2\eta)\epsilon + \eta}
		\\
%
		\frac{h(\fvalL )}{\rvarR} 
		& = e^{\epsilon 
		(\valv_2 - f(a) - 2\eta) - \eta}/e^{\epsilon(\valv_2 - f(a))}
		 =
		e^{ 
		-(2\eta\epsilon + \eta)}
		\end{array}
		\]
		%
		Then, we can derive:
		%
		\[
		\begin{array}{ll}
		|\fvarR - \fvalL|
		& \leq |g(\fvalR ) - h(\fvalL )| \\
		& \leq e^{ 
		(2\eta \epsilon + \eta)}R - 
		e^{-(2\eta\epsilon + \eta)} L \\
		& = L \big(  e^{ \Lambda\epsilon + (2\eta \epsilon + \eta)} 
		- e^{-(2\eta \epsilon + \eta)} \big)\\
		& < (\rvalR - \rvalL)
		e^{(3.45(\epsilon + 1) \eta )},
		\end{array}
		\]
		In the same way, we can derive:
		%
		\[
		\extend{
		|\fvarR{'} - \fvalL{'}|
		\geq 
		|h(\fvalR{'} ) - g(\fvalL{'} )|
		 > e^{-(2\eta \epsilon + \eta)} R 
		 - e^{(2\eta \epsilon + \eta)} L
		> (\rvalR{'} - \rvalL{'})
		e^{(-3.45(\epsilon + 1) \eta)}
		}
		\]
		%
		Both of the bound are derived under the restriction that: $(\epsilon + 1) \eta \leq 1$ and $\epsilon \geq 1$.
		%
		Then we have:
		%
		\[
		\extend{
		e^0 \leq 
		\frac{|\fvalR - \fvalL |}
		{|\fvalR{'} - \fvalL{'}|}
		\leq
		\frac{|g(\fvalR ) - h(\fvalL )|}
		{|h(\fvalR{'} ) - g(\fvalL{'} )|}
		\leq 
		\frac{e^{(3.45(\epsilon + 1) \eta )}(\rvalR - \rvalL) }
		{e^{(-3.45(\epsilon + 1) \eta)} (\rvalR{'} - \rvalL{'})}
		=
		\left(
		e^{6.9( \epsilon + 1) \eta + \epsilon}
		\right).
		}		
		\]
		}
%
	%
	%
\end{proof}


% \begin{proof}
% Induction on $\expr$, we have following cases:

% \end{proof}

\newpage
\bibliographystyle{plain}
\bibliography{verifysnap.bib}



\end{document}















